\section{Electoweak Interaction}

The electromagnetic interaction is described by quantum electrodynamics (QED), which is a quantum field theory that describes the interactions of charged fermions through the exchange of photons. The charged fermions consist of the previously discussed quarks, and the 3 charged leptons, one for each lepton flavour: electron $e^-$, muon $\mu^-$, and tau-lepton $\tau^-$  all with an electric charge of -1. The photon has no electric charge and thus, unlike in QCD, electromagnetic interaction is described only by lepton-gluon couplings. The electromagnetic interaction strength given by the fine-structure constant $\alpha_{em} $ is approximately 1/137 at zero energy scale, QED is thus considered a perturbative theory. $\alpha_{em}$ varies with energy scale, however the variation is limited compared to $\alpha_s$ and increases with increasing energy scale, a effective value of $\alpha_{em} \approx 1/127$ is measured at the Z-boson mass energy scale.\\
Last, the weak interaction with an interaction strength orders lower than the strong and electromagnetic interaction strength is the only interaction which describes interactions between all fermions: quarks, leptons and neutrinos. Neutrinos $\nu$ have no color charge nor electric charge and thus interact only by the weak interaction. There exists three neutrino, one for each lepton generation and are all treated as massless within the SM. However, neutrinos are required to have a non-zero mass in order to explain the observation of neutrino oscilations\cite{de_Salas_2018}. The weak interaction is mediated by either a charged massive $\text{W}^+$/$\text{W}^-$ boson or neutral massive Z boson with a measured measured mass of $80.377 \pm 0.012$GeV and $1.1876 \pm 0.0021$GeV respectively.

In the SM fermions are divided into three generations of fermions:
\begin{itemize}
    \item First generation: electron, electron-neutrino, up quark, and down quark
    \item Second generation: muon, muon-neutrino, charm quark, and strange quark
    \item Third generation: tau-lepton, tau-neutrino, top quark, and bottom quark
\end{itemize}
This division of generations within leptons and neutrinos can be understood based on their weak interaction to the W boson. First, any interaction in the SM conserves electric charge thus the W boson couples only to a lepton and neutrino. In addition the W boson only couples to leptons and neutrinos of the same generation, i.e. a muon and muon-neutrino hence the name muon-neutrino. By contrast the assignment of generations or flavours of quarks has no direct relation to the interaction with the W boson but instead originates from diagonalising Yukawa couplings in the SM Lagrangian. As a result the W boson can couple any up-type quark to any down-type quark, namely mixing of quark generations. The occurrence of flavour changing couplings is nevertheless more rare than flavour conserving couplings as the interaction strength of these couplings are described by the Cabibbo-Kobayashi-Maskaw (CKM) matrix where the value within the CKM matrix $|V_{ab}|$ is multiplied onto the weak interaction strength, i.e. an up quark (u) has a probability of $|V_{us}|^2$ to transition to a strange quark (s) trough weak interaction with a W$^-$ boson with respect to the other down-type quarks. The values of the CKM matrix are measured through the study of weak decays of mesons at high-energy particle colliders such as the LHC. 
\begin{equation}\label{CKM}
    V_{\text{CKM}} = \begin{pmatrix} |V_{ud}| & |V_{us}| & |V_{ub}| \\ |V_{cd}| & |V_{cs}| & |V_{cb}| \\ |V_{td}| & |V_{ts}| & |V_{tb}| \end{pmatrix} = \begin{pmatrix} 0.974 & 0.225 & 0.004 \\ 0.224 & 0.974 & 0.042 \\ 0.009 & 0.041 & 0.999 \end{pmatrix} 
\end{equation}
The Mixing of quark generations trough weak interactions allows for the decay of the more massive higher generation quarks to decay to lighter quarks of the first generation. A unique quark decay is the decay of a top quark which, looking at the CKM matrix, almost exclusively decays to a bottom quark. The bottom quark in turn decays to lighter first and second generation quarks.