\chapter{Overview of Top Physics}

\section{Discovery of the top quark}
The top quark is now known to be the sixth and last quark of the SM. However, the original quark model by GellMann\cite{GELLMANN1964214} and Zweig\cite{Zweig:570209} stated that hadrons consist only of the three lightest quarks: up, down, and strange. In 1963 Cabibbo introduced the idea of electroweak interactions changing the strangeness quantum number of hadrons, which is now called flavour mixing. The existence of flavour mixing consequently predicted the existence of flavour changing neutral current (FCNC), a phenomena unexpectedly not observed in experiments. In an attempt to explain this suppression of FCNCs, Glashow, Iliopoulos, and Maiani\cite{PhysRevD.2.1285} in 1970 proposed the existence of a fourth quark, the charm quark. The charm quark was later discovered in Brookhaven National Laberatory at the $J/\psi$ resonance; this confirmed the existence of two generations of quarks. During this period an experiment by Christenson, Cronin, Fitch, and Turlay\cite{PhysRevLett.13.138} discovered that the  symmetry operation of charge conjugation C and parity P is violated in the weak interaction (CP violation). To make sense of CP violation, Kobayashi and Maskawa in 1973 thought of a mechanism for CP violation through flavor mixing in weak interaction\cite{CKM}. This mechanism required the existence of a third generation of quarks to function and thus the search for a third generation quark began. The discovery of
the $\Upsilon$ resonance at Fermilab\cite{PhysRevLett.39.252} followed shortly after in 1877, a first evidence of the bottom quark. A new question arose whether or not the bottom quark was part of a doublet and if there existed another, at the time undiscovered, quark. Both experimental and theoretical motivated the existence of a sixth quark and with it the search for the top quark began.\\
\\
The initial focus to discover the top quark was at a mass range similar to the bottom quark. Direct searches were performed at the $e^+e^-$ colliders PEP, PETRA, TRISTAN, SLC, and LEP I. Without discovery, the searches placed a lower-limit to the top quark mass of 45GeV. The proton-antiproton collider at CERN with $\sqrt{s}= 540$GeV continued the search with the UA1 experiment and UA2 experiment again without a discovery. They provided lower limits on the top quark mass up to approximately 70GeV. In hindsight with a top quark mass of 173GeV, top quark pair production at the proton-antiproton collider was feasable. However, it was not until the first run of Tevatron, a proton anti-proton collider at the Fermi National Accelerator Laboratory (Fermilab) near Chicago, at $\sqrt{s}=1.8$TeV that the top quark was discovered\cite{RevModPhys.69.137}.