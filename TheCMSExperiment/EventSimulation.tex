\section*{Event Simulation}

To acquire a better understanding of measured data at the CMS experiment as well as to test detector performance and event reconstruction algorithms, collision events are simulated for all processes relevant to the CMS experiment. The simulation of collision events consists out of three seperate steps: matrix element calculation, parton shower simulation, and detector simulation.

The matrix element calculation for most processes is done using the MADGRAPH5\_aMC@NLO program. MADGRAPH5\_aMC@NLO combines the MADGRAPH5\cite{Alwall_2011} algorithm and MC@NLO\cite{frixione2010mcnlo} package to calculate matrix elements and generate events of a process uard events. MADGRAPH5 itself is primarily a program for calculating matrix elements. First the algorithm generates all valid Feynman diagrams for a given process in a model using a diagram generation algorithm. The Feynman diagrams are generated based on the Feynman rules of a given QFT model as input derived by \textit{FeynRules}\cite{Christensen_2009}, a Mathematica package that derives the Feynman rules in a format usefull for MADGRAPH5. The flexibility of choosing the QFT model makes it possible to generate matrix elements for new physics models in addition to the known SM. After the generation of relevant Feynman diagrams, MADGRAPH5 calculates the matrix elements in LO and NLO of these Feynman diagrams. Matrix elements in LO and NLO in QCD corrections of a process consist of several three-level and one-loop amplitudes, the MADGRAPH5 program is thus built on computing tree-level and one-loop amplitudes for processes. By combining all matrix elements in LO and NLO MADGRAPH5 can provide the cross-section of the desired process up to NLO. Last, the event generation of MADGRAPH5\_aMC@NLO is done by the MC@NLO\cite{frixione2010mcnlo} package using the provided matrix elements. MC@NLO is based on a Monte Carlo generator\cite{MC} and provides events containing the information on the particles of the process without including radiation or parton showering.\\
\\
\textbf{Parton shower simulation} is achieved by the Pythia8\cite{Sj_strand_2015} program. Pythia8 is intended to describe initial and final state radiation (ISR and FSR)...\\
\\
Last, for the \textbf{detector simulation} events go trough the Geant4\cite{AGOSTINELLI2003250} toolkit. Geant4 at its core simulates the interaction of particles with matter while traversing the detector.